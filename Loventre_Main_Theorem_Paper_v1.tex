\documentclass[11pt,a4paper]{article}

\usepackage[utf8]{inputenc}
\usepackage[T1]{fontenc}
\usepackage[italian]{babel}
\usepackage{amsmath,amssymb,amsthm}
\usepackage{geometry}
\usepackage{hyperref}
\geometry{margin=2.5cm}

\title{Il Teorema di Loventre v5\\
\large SAFE Witness, Curvatura Informazionale e Asimmetria verso Black Hole}
\author{Vincenzo Loventre}
\date{\today}

\theoremstyle{plain}
\newtheorem{theorem}{Teorema}
\newtheorem{lemma}{Lemma}
\newtheorem{proposition}{Proposizione}
\theoremstyle{definition}
\newtheorem{definition}{Definizione}
\theoremstyle{remark}
\newtheorem{remark}{Osservazione}

\begin{document}

\maketitle

\begin{abstract}
In questo lavoro introduciamo e formalizziamo, in Coq, una piccola ma
rigida teoria di \emph{curvatura informazionale} per classi di problemi
computazionali astratti. Il risultato principale, il Teorema di
Loventre v5, afferma che esiste un witness concreto in una classe
``stabile'' (denotata $P\_{\mathrm{STR}}$) che soddisfa una condizione
di sicurezza globale (\emph{SAFE}), e che allo stesso tempo la
curvatura informazionale verso una classe ``black-hole''
($P\_{\mathrm{BH}}$) presenta una asimmetria strutturale rispetto a una
classe intermedia ``accessibile'' ($P\_{\mathrm{ACC}}$). Il teorema è
dimostrato completamente in Coq, con un ponte operativo verso Python
tramite un export JSON che verifica computazionalmente le stesse
relazioni di curvatura.
\end{abstract}

\section{Introduzione}

L'obiettivo di questo lavoro è introdurre una struttura minimale, ma
formalmente solida, che colleghi:
\begin{itemize}
  \item la presenza di un \emph{witness} concreto in una classe ``buona''
        (denotata $P\_{\mathrm{STR}}$),
  \item una proprietà globale di sicurezza (\emph{SAFE}),
  \item una nozione discreta di \emph{curvatura informazionale}
        tra classi,
  \item un'asimmetria strutturale nella transizione verso una classe
        \emph{black-hole} $P\_{\mathrm{BH}}$.
\end{itemize}

Il cuore concettuale è che, pur non parlando direttamente di problemi
classici come SAT o TSP, la geometria informazionale che definiamo
riflette un principio di \emph{irreversibilità}: la classe
$P\_{\mathrm{BH}}$ si comporta come un attrattore assorbente nello
spazio delle configurazioni informazionali, mentre $P\_{\mathrm{STR}}$
resta ancorata a una regione SAFE.

Tutto il nucleo del risultato è stato formalizzato in Coq
(versione 8.18) in una catena di moduli che vanno da un witness
concreto, attraverso un predicato di sicurezza, fino a un teorema
principale chiamato \texttt{Loventre\_Main\_Theorem}.

\section{Struttura formale delle classi}

Introduciamo tre classi astratte di configurazioni informazionali:

\begin{definition}[Classi informazionali]
Sia $\mathcal{C} = \{P\_{\mathrm{STR}}, P\_{\mathrm{ACC}}, P\_{\mathrm{BH}}\}$
un insieme di tre classi astratte:
\begin{itemize}
  \item $P\_{\mathrm{STR}}$: una classe ``stabile'',
  \item $P\_{\mathrm{ACC}}$: una classe intermedia ``accessibile'',
  \item $P\_{\mathrm{BH}}$: una classe di tipo \emph{black-hole}.
\end{itemize}
In Coq queste sono realizzate come costruttori di un tipo induttivo
\texttt{LClass\_v3}.
\end{definition}

\section{Curvatura informazionale e differenza di curvatura}

Definiamo una funzione di curvatura discreta
$\kappa : \mathcal{C} \to \mathbb{N}$ e una differenza di curvatura
$\Delta \kappa$.

\begin{definition}[Curvatura]
Una \emph{curvatura informazionale} è una mappa
$\kappa: \mathcal{C} \to \mathbb{N}$. Nel modello implementato in Coq
(\\\texttt{Loventre\_v3\_Curvature.v}) si assegna ad ogni classe un
valore naturale, ad esempio
\[
  \kappa(P\_{\mathrm{STR}}) = 0,\quad
  \kappa(P\_{\mathrm{ACC}}) = 1,\quad
  \kappa(P\_{\mathrm{BH}}) = 2.
\]
\end{definition}

\begin{definition}[Differenza di curvatura]
Data la curvatura $\kappa$, definiamo la differenza di curvatura
$\Delta\kappa : \mathcal{C} \times \mathcal{C} \to \mathbb{N}$ come
\[
  \Delta\kappa(c\_1, c\_2) = \max\{\,\kappa(c\_2) - \kappa(c\_1),\,0\,\}.
\]
Nel codice Coq (\texttt{Loventre\_v3\_DeltaCurvature.v}) questa è
realizzata come differenza su numeri naturali con taglio a zero.
\end{definition}

Il lemma di asimmetria fondamentale è:

\begin{lemma}[Asimmetria di curvatura v3]
\label{lem:asymmetry}
In Coq è dimostrato il lemma
\texttt{Loventre\_v3\_asymmetry\_final}, che corrisponde all'enunciato:
\[
  1 + \Delta\kappa(P\_{\mathrm{ACC}}, P\_{\mathrm{BH}})
  \;\leq\;
  \Delta\kappa(P\_{\mathrm{STR}}, P\_{\mathrm{BH}}).
\]
\end{lemma}

Questo implica che, verso $P\_{\mathrm{BH}}$, la ``distanza di
curvatura'' di $P\_{\mathrm{STR}}$ è strutturalmente più ampia di
quella di $P\_{\mathrm{ACC}}$.

\section{Witness concreto e predicato SAFE}

Sul lato ``v2'' della teoria, costruiamo un witness concreto che vive
nella classe $P\_{\mathrm{STR}}$ e definiamo un predicato di sicurezza
globale.

\begin{definition}[Witness e predicato SAFE]
Nel modulo \texttt{Loventre\_Witness\_Loader.v} definiamo una istanza
concreta \texttt{LoventreWitness\_Instance} con un campo
\texttt{w\_lmetrics\_type} che, nel caso specifico, vale la stringa
\texttt{"P\_STR"}. Nel modulo
\texttt{Loventre\_SAFE\_Predicate.v} viene definito un predicato
\[
  \mathrm{SAFE} : \mathcal{C} \to \mathrm{Prop}
\]
e, in particolare, un costruttore
\[
  \mathrm{Safe\_P\_STR} : \mathrm{SAFE}(P\_{\mathrm{STR}}).
\]
Nel modulo \texttt{Loventre\_Witness\_SAFE\_Global.v} si combina il
witness concreto con il predicato SAFE ottenendo il lemma
\[
  \mathrm{Witness\_is\_SAFE} : \mathrm{SAFE}(P\_{\mathrm{STR}}).
\]
\end{definition}

\section{Unificazione e Teorema Principale}

Lo strato v4 unifica la parte SAFE (v2) e la parte di curvatura
(asimmetria v3) in un unico lemma, che è la base del teorema globale.

\begin{definition}[Sistema SAFE e asimmetria valida]
Nel modulo \texttt{Loventre\_v4\_Unification.v} definiamo due proposizioni:
\[
  \mathrm{SAFE\_sys} := \mathrm{SAFE}(P\_{\mathrm{STR}}),
\]
\[
  \mathrm{ASYM} := 1 + \Delta\kappa(P\_{\mathrm{ACC}}, P\_{\mathrm{BH}})
     \leq \Delta\kappa(P\_{\mathrm{STR}}, P\_{\mathrm{BH}}).
\]
\end{definition}

In Coq queste corrispondono a \texttt{Loventre\_v4\_system\_is\_SAFE}
e \texttt{Loventre\_v4\_asymmetry\_valid}. Il lemma di unificazione è:

\begin{lemma}[Unificazione v4]
Nel modulo v4 è dimostrato il lemma:
\[
  \mathrm{SAFE\_sys} \wedge \mathrm{ASYM}.
\]
In Coq: \texttt{Loventre\_v4\_unified}.
\end{lemma}

Questo lemma è poi incapsulato nel seguente teorema principale:

\begin{theorem}[Teorema di Loventre v5]
\label{thm:main}
Il teorema principale, formalizzato in Coq nel modulo
\texttt{Loventre\_Main\_Theorem.v}, afferma:
\[
  \mathrm{SAFE}(P\_{\mathrm{STR}}) \;\wedge\;
  \Bigl(
    1 + \Delta\kappa(P\_{\mathrm{ACC}}, P\_{\mathrm{BH}})
    \leq \Delta\kappa(P\_{\mathrm{STR}}, P\_{\mathrm{BH}})
  \Bigr).
\]
\end{theorem}

\begin{proof}[Idea della dimostrazione]
La dimostrazione in Coq è molto compatta perché si appoggia sui lemmi
combinati nello strato v4. In particolare:
\begin{itemize}
  \item dalla parte v2 (witness + SAFE) si ottiene
        $\mathrm{SAFE}(P\_{\mathrm{STR}})$;
  \item dalla parte v3 (curvatura + asimmetria) si ottiene il lemma
        \texttt{Loventre\_v3\_asymmetry\_final} che viene riallineato
        in v4 come proposizione $\mathrm{ASYM}$;
  \item il lemma \texttt{Loventre\_v4\_unified} combina queste due
        componenti in un'unica congiunzione, che viene poi incapsulata
        in \texttt{Loventre\_Main\_Theorem}.
\end{itemize}
\end{proof}

\section{Ponte Coq--Python via JSON}

Un aspetto importante del progetto è l'esistenza di un ponte operativo
con il motore Python: una struttura dati che rappresenta lo stato v3
(distanze di curvatura e asimmetria) viene serializzata in JSON e
ricaricata in Python, dove vengono verificati gli stessi vincoli
numerici (ad esempio $\Delta\kappa(P\_{\mathrm{ACC}}, P\_{\mathrm{BH}})
< \Delta\kappa(P\_{\mathrm{STR}}, P\_{\mathrm{BH}})$). Questo fornisce
una forma di ``doppia validazione'':
\begin{itemize}
  \item lato Coq: prova formale del teorema,
  \item lato Python: verifica computazionale dei parametri esportati.
\end{itemize}

\section{Conclusioni e sviluppi futuri}

Il Teorema di Loventre v5 rappresenta una prima formalizzazione di
un principio di \emph{irreversibilità informazionale} tra classi
astratte di problemi:

\begin{itemize}
  \item esiste un witness concreto e SAFE in una classe stabile
        $P\_{\mathrm{STR}}$;
  \item la curvatura informazionale verso $P\_{\mathrm{BH}}$ è
        asimmetrica e ``più distante'' per $P\_{\mathrm{STR}}$
        rispetto a $P\_{\mathrm{ACC}}$;
  \item l'intero risultato è formalizzato in Coq, con un ponte attivo
        verso una implementazione Python.
\end{itemize}

Sviluppi futuri includono:
\begin{itemize}
  \item una formalizzazione più ricca della nozione di entropia
        informazionale e potenziale,
  \item il collegamento esplicito con istanze di problemi combinatoriali
        (ad esempio SAT critico, TSP critico),
  \item l'estensione del framework a classi multiple e strutture di
        tipo ``barriera SAFE'' più complesse.
\end{itemize}

\end{document}

