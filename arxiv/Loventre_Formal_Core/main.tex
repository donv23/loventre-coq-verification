\documentclass[11pt]{article}

\usepackage{amsmath,amssymb,amsthm}
\usepackage{hyperref}
\usepackage{geometry}
\geometry{margin=1in}

\title{The Loventre Formal Core:\\
An Abstract Geometric Model of Structural Separation}
\author{Vincenzo Loventre}
\date{}

\newtheorem{definition}{Definition}
\newtheorem{lemma}{Lemma}
\newtheorem{theorem}{Theorem}

\begin{document}
\maketitle

\section*{Scope and Non-Claims}
This paper presents the \emph{formal core} of the Loventre Model.
All results are internal to the model.
No claim is made regarding the classical \emph{P vs NP} problem or any external complexity-theoretic conjecture.

\section{Abstract Computational Structure}

\begin{definition}[Computational Space]
Let $\mathcal{C}$ be a non-empty set whose elements are called computations.
\end{definition}

\begin{definition}[Structural Values]
Let $\mathcal{V}$ be an abstract set equipped with a relation $\leq$.
No algebraic or order-theoretic properties are assumed.
\end{definition}

\begin{definition}[Admissible Transitions]
Let $\mathcal{R} \subseteq \mathcal{C} \times \mathcal{C}$ be a relation of admissible transitions.
\end{definition}

\section{Structural Invariants}

Let
\[
\kappa, H, \chi : \mathcal{C} \to \mathcal{V}
\]
be structural invariants.

\begin{definition}[Structural Variation]
Let $\Delta(f,c_1,c_2)$ denote the structural variation of invariant $f$ between $c_1$ and $c_2$.
\end{definition}

\section{Efficiency}

\begin{definition}[Efficient Trajectory]
A trajectory $\tau \subseteq \mathcal{C}$ is said to be efficient if there exists $B \in \mathcal{V}$ such that
for all invariants $f \in \{\kappa,H,\chi\}$ and all $c_1,c_2 \in \tau$,
\[
\Delta(f,c_1,c_2) \leq B.
\]
\end{definition}

\section{Structural Horizons}

\begin{definition}[Horizon]
A computation $c \in \mathcal{C}$ exhibits a structural horizon if for every trajectory $\tau$ containing $c$
and reaching a target region $\mathcal{T}$, the trajectory is not efficient.
\end{definition}

\section{Structural Separation}

\begin{theorem}[Efficiency Obstruction]
No efficient trajectory starting at a computation exhibiting a structural horizon can reach the target region.
\end{theorem}

\begin{proof}
Immediate from the definition of structural horizon.
\end{proof}

\section*{Formal Verification}
The above definitions and theorem are mechanically verified in Coq in the module
\texttt{02\_Advanced/Geometry/Loventre\_Formal\_Core.v}.
\end{document}

