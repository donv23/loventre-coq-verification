\documentclass[11pt]{article}

\usepackage{amsmath,amssymb}
\usepackage{hyperref}
\usepackage{geometry}
\geometry{margin=1in}

\title{An Interpretative Layer for the Loventre Formal Core}
\author{Vincenzo Loventre}
\date{}

\begin{document}
\maketitle

\section*{Purpose and Scope}

This document provides an \emph{interpretative reading} of the Loventre Formal Core.
It introduces no new axioms, definitions, or theorems.
All formal results are contained exclusively in the Formal Core and its Coq verification.

No claim is made regarding the classical \emph{P vs NP} problem.

\section{From Structure to Interpretation}

The Loventre Formal Core defines an abstract notion of:
\begin{itemize}
  \item efficient trajectories,
  \item structural horizons,
  \item obstruction of efficiency.
\end{itemize}

These notions are purely formal.
This document proposes an interpretation of these notions in terms of
\emph{computational phases}.

\section{Phase Interpretation}

We introduce the following \emph{interpretative terminology}.

\subsection*{P-like Phase (Interpretative)}

A computation is said to be \emph{P-like} if it belongs to a region of the computational space
where no structural horizon is present.
In this phase, efficient trajectories are structurally admissible.

\subsection*{NP-like-Black-Hole Phase (Interpretative)}

A computation is said to be \emph{NP-like-black-hole} if it exhibits a structural horizon.
In this phase, efficient trajectories are structurally obstructed.

These terms are purely interpretative and do not correspond to classical complexity classes.

\section{Algorithmic Intuition}

From an algorithmic perspective, the obstruction of efficiency may manifest as:
\begin{itemize}
  \item exponential or super-polynomial growth of search effort,
  \item loss of global information accessibility,
  \item saturation of heuristic guidance.
\end{itemize}

These phenomena are not assumed in the Formal Core;
they are \emph{explained} by it, under the proposed interpretation.

\section{On Apparent Counterexamples}

Occasional algorithmic successes in horizon-bearing regions
may be interpreted as rare, non-scalable events.
Such events do not contradict the Formal Core,
as they do not correspond to efficient trajectories.

\section{Relation to the Formal Core}

All interpretative claims in this document are grounded in the following correspondence:

\begin{center}
\begin{tabular}{ll}
\textbf{Formal Core} & \textbf{Interpretation} \\
\hline
Efficient trajectory & Feasible algorithmic evolution \\
Structural horizon & Intrinsic computational barrier \\
Efficiency obstruction & Phase separation \\
\end{tabular}
\end{center}

The validity of the Formal Core does not depend on this interpretation.

\section*{Conclusion}

The Loventre Formal Core admits a natural interpretation
in terms of computational phases and intrinsic barriers.
This interpretation is optional, external, and non-binding.

The Formal Core remains valid independently of any interpretative layer.

\end{document}

